%!TEX encoding = UTF-8 Unicode
%preamble
\documentclass{article}
\usepackage{titlesec}
\usepackage[top=1in, left=1in]{geometry}
\usepackage{fancyhdr}
%fancy header
    \pagestyle{fancy}
    \lhead{\textsc{Cosi} 140B}
    \chead{\textbf{MojiSem: Semantic and Discourse functions of Emoji}}
    \rhead{\today}
    \lfoot{}
    \cfoot{}
    \rfoot{}

%template fillers
\newcommand{\groupmember}[1]{#1}
\newcommand{\role}[1]{#1}
%signature lines
\newcommand{\sigline}[2]{\vspace{2em} #1 \hfill \line(1,0){#2}}

%document
\begin{document}

\section{Group Members} % (fold)
\label{par:group_members}
This document represents an agreement between the following parties:
\begin{itemize}
    \item \groupmember{Orion Montoya}
    \item \groupmember{Noa Naaman}
    \item \groupmember{Hannah Provenza}
\end{itemize}
% paragraph group_members (end)

\section{Goals} % (fold)
\label{par:expectations}
Our expectations for the course project are as follows:
\begin{enumerate}
    \item Evaluate reliability of existing Twitter POS taggers on current emoji.
    \item Collect corpus of emoji-bearing tweets; ideally containing as many emoji characters as possible.
    \item Potentially: limit corpus to texts that can be identified as English.
    \item Specify useful judgments with some hope of attaining IAA.
    \item Annotate non-taggable/wrongly-tagged emoji characters by semantic or discourse function.
    \item Produce summary measures of the diversity of emoji usages: polysemy, typical functions.
    \item Predict the function or meaning of a sequence of emoji characters in unseen text.
\end{enumerate}
% paragraph expectations (end)

\section{Skills} % (fold)
\label{sec:skills}

The relevant skills that each member believes will aid in their contributions are as follows:

\begin{description}
    \item[\groupmember{Orion}:] Documentation, code management, corpus development and analysis
    \item[\groupmember{Noa}:] Annotation spec, semantic and pragmatic theories
    \item[\groupmember{Hannah}:] Corpus development, coding lead
\end{description}
% section skills (end)

\section{Roles} % (fold)
\label{sec:roles}

Roles within the group will be designated as follows:

\begin{description}
    \item[\role{Orion}:] Documentation, code management, corpus development and analysis
    \item[\role{Noa}:] Annotation spec, semantic and pragmatic theories
    \item[\role{Hannah}:] Corpus development, coding lead
\end{description}
% section roles (end)

\section{Agreement} % (fold)
\label{par:agreement}
The undersigned hereby agree, to the best of their respective abilities and skills
as outlined in Section \ref{sec:skills}, to contribute to the project according to
their roles as designated in Section \ref{sec:roles}.

\textit{I solemnly swear that I am up to no good.}
%\footnote{\emph{Disclaimer}: This is a preliminary group contract. As our understanding of the project becomes more clear we reserve the option to revise and resubmit a modified contract to reflect that understanding.}
% paragraph agreement (end)
\vspace{2em}\\
\sigline{\textbf{\groupmember{Orion Montoya}}}{300}\\
\sigline{\textbf{\groupmember{Noa Naaman}}}{300}\\
\sigline{\textbf{\groupmember{Hannah Provenza}}}{300}\\
\end{document}
