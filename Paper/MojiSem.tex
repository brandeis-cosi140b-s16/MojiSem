%
% Based on File acl2015.tex
%

\documentclass[11pt]{article}
\usepackage{acl2015}
\usepackage{times}
\usepackage{url}
\usepackage{latexsym}
\usepackage{gb4e}

\title{Varying linguistic purposes of emoji in (Twitter) context}

\author{Noa Na'aman, Hannah Provenza, Orion Montoya\\
  Brandeis University \\
  {\tt \{nnaaman,hprovenza,obm\}@brandeis.edu}\\}

\date{}

\begin{document}
\maketitle

\begin{abstract}

Research into emoji in textual communication has, thus far, focused on high-frequency usages and
the ambiguity of interpretations. Investigation of emoji uses across a wide range of uses can divide them
into different linguistic functions: function and content words, or multimodal affective markers. We report on an
annotation task on English Twitter data with the goal of classifying emoji usage by these categories, and on the
effectiveness of a classifier trained on these annotations. We find that 
PART OF IT IS STRAIGHTFORWARD, 
but
ANOTHER PART OF IT IS COMPLICATED.

\end{abstract}

\section{Background}

Early work on Twitter emoticons \cite{tylerEmoticons2012} pre-dated the wide spread of Unicode emoji on mobile and desktop devices. Schnoebelen studied the 
Recent work \cite{MillerEmoji2016} has explored the 



\section{Annotation task}

\subsection{Data collection and filtering}
Tweets were pulled from the public Twitter streaming API using the \texttt{tweepy} library. The collected tweets were automatically filtered to include only tweets with characters from the Emoji Unicode ranges (i.e. generally U+1FXXX, U+26XX--U+27BF); only tweets labeled as being in English; to exclude tweets with embedded images or links (more below).

Tweets with links and images were excluded from consideration to reduce time investment and cognitive load for annotators. Our early explorations found frequent cases where emoji were tweeted to show a reaction to an attached image or linked page (especially a blog post or news story) and that these tended toward ambiguous interpretations akin to those found by Miller et al. A given tweet's U+1F62D `loudly crying face' might be showing true sympathy, or sarcastically saying ``cry my a river.'' The amount of annotator effort necessary for an annotator to determine this in context would require an understanding of the tweeter's past opinions and their stance on the parties involved in the story they linked. These are very interesting questions for future research, but we determined them to be out of scope for the present research, which focuses on high-level, coarse-grained distinctions.

Redundant/duplicate tweets were filtered by comparing tweet texts after removal of hashtags and @mentions; this left only a small number of cloned duplicates.

\bibliographystyle{acl}
\bibliography{MojiSem}


\end{document}
