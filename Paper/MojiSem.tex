%
% Based on File acl2015.tex
%

\documentclass[11pt]{article}
\usepackage{acl2015}
\usepackage{times}
\usepackage{url}
\usepackage{latexsym}
\usepackage{gb4e}

\title{Varying linguistic purposes of emoji in (Twitter) context}

\author{Noa Na'aman, Hannah Provenza, Orion Montoya\\
  Brandeis University \\
  {\tt \{nnaaman,hprovenza,obm\}@brandeis.edu}\\}

\date{}
\def\st{$\bullet$}

\begin{document}
\maketitle

\begin{abstract}

Research into emoji in textual communication has, thus far, focused on high-frequency usages and
the ambiguity of interpretations. Investigation of emoji uses across a wide range of uses can divide them
into different linguistic functions: function and content words, or multimodal affective markers. We report on an
annotation task on English Twitter data with the goal of classifying emoji usage by these categories, and on the
effectiveness of a classifier trained on these annotations. We find that 
PART OF IT IS STRAIGHTFORWARD, 
but
ANOTHER PART OF IT IS COMPLICATED.

\end{abstract}

\section{Background}

Early work on Twitter emoticons \cite{tylerEmoticons2012} pre-dated the wide spread of Unicode emoji on mobile and desktop devices. Schnoebelen studied the 
Recent work \cite{MillerEmoji2016} has explored the 



\section{Annotation task}

\subsection{Guidelines}

Although recognizing the presence of emoji characters is trivial and unambiguous, the linguistic distinctions
we sought to annotate were ambiguous and seemed prone to disagreement. Therefore in the guidelines we
structured the annotation process in such a way as to minimize cognitive load and lead the annotators to
intuitive and natural decisions. Our task was motivated partially by the observation that emoji are used
in contexts that make them graphical replacements that map to the existing lexicon, and are therefore
straightforward to interpret. The guidelines build on this observation and dictate a process that aims
to take advantage of such uses. The process was a flow that presented annotators with a few simple questions
at each step, to determine whether to assign a label or to move on to the next step.

Our guidelines gave a cursory background about emoji and their uses in social media, assuming no particular
familiarity with the range of creative uses of emoji. In hindsight we realized that we assumed that the
annotators would have a fair degree of familiarity with Twitter. The short-message social
platform has many distinctive cultural and communicative codes of its own, not to mention subcultures, and
continuously evolving trends combined with a long memory. As two of the authors are active and engaged
users of Twitter, we unfortunately took it for granted that our annotators would be able to decipher emoji
in contexts that required knowledge of InterNet language and Twitter norms.

The task included:

%\begin{itemize}
%\item 
%\end{itemize}


\begin{itemize}
  \item  Identifying each emoji in the tweet 
  \item  Deciding whether multiple contiguous emoji should be considered separately or as a group
  \item  Choosing the right tag for the emoji (or sequence)
  \item  Providing a translation or interpretation for each tagged span.
\end{itemize}

Eliciting an interpretation serves two main goals. First, as a coercive prompt for the user to bias them
toward a linguistic interpretation. A replaceable phrase that fits with the grammar of the sentence is a
different proposition than a marker that amounts to a standalone utterance such as ``I am laughing'' or ``I am sad''.
Secondly, one of the eventual applications of annotated corpus may be emoji-sense disambiguation (ESD), and mapping
to a lexicalized expression would be useful grounding for future ESD tasks. The text field was very helpful
during the adjudication process, clarifying the annotators' judgments and understanding of the task (when
done correctly).

For each tweet, annotators were asked to do a first read without annotating anything, to get a sense of the
general message of the tweet and to think about the relationship between the emoji and the text. After the
first reading, they are asked to determine whether the emoji is serving as punctuation or a function word;
then if it is a content word; and if it is neither of those, then to examine it as a multimodal emoji.
A key test, in our opinion, was asking annotators to simulate reading the message of the tweet aloud to
another person.  If a listeners comprension of the core message seemed to require a word or phrase to be spoken
in place of an emoji, then that would be a compelling sign that it should be tagged as function or content.

Multimodal was the category assigned to uses that failed the first two tests. We provided
some guidance for deciding between `topic', `attitude' or `gesture' as sub-types of the multimodal category.
During adjudication it became obvious that we left this part too open to interpretation. We encountered many
cases where we had trouble deciding which of two disagreeing annotators was `right' about choosing one type
over another. A clearer typology of multimodal emojis, and, if possible, a more deterministic procedure for
labeling emoji with these subtypes, are the major desiderata for any future iteration of this project.

\subsubsection{Later amendments to the guidelines}

After the first batch of annotation work, in response to concerns and suggestions of our annotators, we made
three major changes to the guidelines:
\begin{itemize}
  \item Added the option of tagging a tweet as ``out of scope,'' since some uses did not fit in our task description. One
major example was an entire movie plot retold as a sequence of emoji characters. This is out of scope because
here the emoji do not appear in a context that puts them in relation to other text, which is the phenomenon
we sought to examine.

\item Added co-reference tagging for topic-marker emoji that had overt co-referring words in the tweet. These are
the only cases where we tagged non-emoji character spans.

\item Added the option of tagging sequences of emoji that were split up by non-emoji text but should still be
treated as one unit. This wasn't used at all in the following annotation batches, because the tweet format that
made it necessary did not appear in the tweet sets for those batches.
\end{itemize}

\subsection{Data collection and filtering}
Tweets were pulled from the public Twitter streaming API using the \texttt{tweepy} library. The collected tweets were automatically filtered to include only tweets with characters from the Emoji Unicode ranges (i.e. generally \texttt{U+1FXXX}, \texttt{U+26XX}--\texttt{U+27BF}); only tweets labeled as being in English; to exclude tweets with embedded images or links (more below).

Tweets with links and images were excluded from consideration to reduce time investment and cognitive load for annotators. Our early explorations found frequent cases where emoji were tweeted to show a reaction to an attached image or linked page (especially a blog post or news story) and that these tended toward ambiguous interpretations akin to those found by Miller et al. A given tweet's \texttt{U+1F62D `loudly crying face'} might be showing true sympathy, or sarcastically saying ``cry my a river.'' The amount of annotator effort necessary for an annotator to determine this in context would require an understanding of the tweeter's past opinions and their stance on the parties involved in the story they linked. These are very interesting questions for future research, but we determined them to be out of scope for the present research, which focuses on high-level, coarse-grained distinctions.

Redundant/duplicate tweets were filtered by comparing tweet texts after removal of hashtags and @mentions; this left only a small number of cloned duplicates.

\subsection{Inter-annotator agreement}

We anticipated strong agreement on the marking of emoji spans. But annotators were encouraged to group multiple emoji in a single span if they were a semantic/functional unit.
This exposed some differences of opinion, with one annotator tagging a pair of characters together, another separately. So the overall Krippendorff $\alpha$ for annotations was around 0.9. 

The assignment of category labels was a bit trickier, especially given the subcategory labels. See table \ref{tab:IAA}.

\begin{table}[htbp]
  \centering
  \begin{tabular}{@{} |lccccc| @{}}
    \hline
    Dataset & \# taters & span rem & total & mm & content \\
    \hline
    Week 1     & 4 & 78 & 0.2071  & 0.4251  & 0.1311  \\
    Week 2     & 4 & 49 & 0.8743  & 0.7158  & 0.8531  \\
    Week 3A    & 2 & 11 & 0.9096  & 0.4616  & 0.792  \\
    Week 3B    & 2 & 6  & 0.7436  & 0.3905  & 1.0  \\
    Week 4A    & 2 & 3  & 0.8789  & 0.4838  & 0.7435  \\
    Week 4B    & 2 & 1  & 0.3954  & 0.5078  & 1.0  \\
    \hline
    Total/mean & 4 & 150 & 0.6681 & 0.4974  & 0.7533  \\
    \hline
  \end{tabular}
  \caption{Fleiss's $\kappa$ scores and other annotation numbers}
  \label{tab:IAA}
\end{table}

\section{Machine-learning task}

Our experiment was to study 

\subsection{Data sparseness}

For reasons described above, we had a small and arbitrary sample of emoji usage available to study. Although we tagged 775 spans in 567 tweets, we only saw 300 distinct emoji, and 135 of them occurred only once. This drove us away from
We felt that needed more complex features than we could get with an HMM, and we needed our experiments to 
control for the contribution of the individual character vs. the other features. So we went with Conditional Random Fields \cite{CRFcite}.


\subsection{Feature engineering}

We used CRFSuite \cite{CRFSuite} and, after experimenting with the different algorithms available,
chose Averaged Perceptron \cite{AvePerceptron} for its superior results. See table \ref{tab:PRF1}

\begin{table}[htp]
\caption{Performance of feature iterations}
\begin{center}
\begin{tabular}{|l| c c c c c c|}
\hline
feature                 & F1 word   & F1 mm  & P cont & R cont   & F1 cont & Macro-avg F1 \\\hline
character               & 0.9721    & 0.7481 & 0.3571 & 0.3333   & 0.3448  & 0.8441\\
 prev +emo?         & 0.9914    & 0.8649 & 0.4286 & 0.4000   & 0.4000  & 0.8783\\
 prev +POS              & 0.9914    & 0.8784 & 0.5000 & 0.4667   & 0.4828  & 0.8921\\
 prev +position         & 0.9914    & 0.8844 & 0.4667 & 0.4667   & 0.4667  & 0.9028\\
 prev +contexty         & 0.9914    & 0.8831 & 0.6250 & 0.3333   & 0.4348  & 0.8848\\
prev +emo\_class (best?)& 0.9914    & \textbf{0.8933} & \textbf{0.7273} & 0.5333   & 0.6154  & \textbf{0.9168}\\
best - character        & 0.9906    & 0.8514 & 0.6429 & \textbf{0.6000}   & 0.6207  & 0.9090\\
best - contexty         & 0.9922    & 0.8750 & 0.4706 & 0.5333   & 0.5000  & 0.8945\\
emo?+POS+emo\_class & 0.9914 & 0.8421 & 0.6000 & 0.4000 & 0.4800 & 0.8855\\
\hline
\end{tabular}
\end{center}
\label{tab:PRF1}
\end{table}%

A standard HMM would have treated observations of the 135 nonce emoji as entirely unique and incomparable. The best thing we could likely have done with an HMM is shown on the final line of table \ref{tab:PRF1}: just using a tuple of \texttt{< is\_emoji, POS, emoji\_class >}, ignoring the individual character, yields performance that is competitive with the early iterations of the CRF feature set.


\section{Future directions}

We clearly need more data. 89 examples of content and functional uses of emoji is not enough to reliably model the behavior of these characters. Since our
classification is mostly binary. More annotation may yield much richer models of the variety, and will help get a better handle on the range of emoji polysemy.
Clustering of contents based on observed features may help induce more empirically valid subtypes than the ones defined by our specification.

This project was motivated by a curiosity, and a hypothesis, about emoji semantics. Anglophone Twitter users use emoji in their tweets
for a wide range of purposes, and a given emoji character means different things in different contexts. It seems inevitable that emoji will
become subject to lexicographical description. Modern lexicography has evolved from millennia of attention to words and their combinations.
Every emoji linguist notes the fascinating range of pragmatic and multi-modal effects that emoji can have in electronic communication. If
these effects are to be given lexicographical treatment and categorization, they must also be organized into categories that are not part of
the typical range of classes used to talk about words.



We have mentioned the notion of emoji-sense disambiguation (ESD). ESD would require an empirical inventory of emoji senses, presumably
from an empirical lexicon of this sort. 

Even our small sample has shown a number of characters that are genuinely used both as content words
and as topical or gestural cues. Sometimes emoji of the ``Earth globe'' (\texttt{U+1F30D}--\texttt{U+1F30F}) are used to mean `world', 
and sometimes they simply decorate a tweet that is expressing an environmental message. Sometimes these three globes---one showing Earth centered
on Europe/Africa, one showing Earth centered on the Americas, and one centered on Asia/Australia---are used to denote those specific regions, 
or sometimes they are concatenated, either for emphasis (`the whole entire world') or to indicate the passing of time. There are a number of flower emoji, and sometimes they are used to decorate a message about flowers themselves, and sometimes they add sentiment to a message---and, just as in culture away from keyboards, a rose \texttt{U+1F339} conveys a different sentiment than a sunflower \texttt{U+1F33B}.


\bibliographystyle{acl}
\bibliography{MojiSem}


\end{document}
